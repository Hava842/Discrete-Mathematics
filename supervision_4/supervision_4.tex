\documentclass{exam}
\usepackage[utf8]{inputenc}
\usepackage{amsmath}
\usepackage{hyperref}
\usepackage{enumitem}
\usepackage{ dsfont }
 
\begin{document}
\noindent
\large\textbf{Discrete Mathematics 4} \hfill Supervisior: Marton Havasi \\
\normalsize Lectures 10-12 \hfill 04/12/2017

\paragraph{Core questions (everyone is expected to solve these exercises)}
\begin{questions}
\question State the Reflexivity, Transitivity and Antisymmetry properties of the subset ($\subseteq$) operator.

\question What is the number of divisors of $x=2\cdot 3\cdot 5\cdot 7\cdot 11\cdot 13 \cdot 17 \cdot 19$ ?

\question 2015 Paper 2 Question 9 part b, \href{http://www.cl.cam.ac.uk/teaching/exams/pastpapers/y2015p2q9.pdf}{Link}

\question Exercise sheet 4.1.2 

Prove that, for any positive integer $n$, a $2^n$ by $2^n$
square grid with any one square removed can be tiled
with L-shaped pieces consisting of 3 squares.

\question Prove, using mathematical induction, that $n^3 < 3^n$ for $n \in \mathds{N}$.

\question Prove that for every positive integer $n$ there exists an $n$ digit number divisible by $5^n$ all of whose digits are odd. 

(Hint: using induction on $n$. If $a_1a_2 \dots a_n$ is divisible by $5^n$ show that one of $1a_1a_2\dots a_n$, $3a_1a_2\dots a_n$, $5a_1a_2\dots a_n$, $7a_1a_2\dots a_n$, $9a_1a_2\dots a_n$ is divisible by $5^{n+1}$.) \footnote{Source: http://web.mat.bham.ac.uk/R.W.Kaye/}

\end{questions}

\paragraph{Tryhard questions (recommended)}
\begin{questions}

\question Excercise sheet 4.3.2 The set of (univariate) polynomials (over the rationals) on a variable $x$ is defined as that of arithmetic
expressions equal to those of the form $\sum_{i=0}^n a_ix^i$, for some $n \in \mathcal{N}$ and some $a_1, \dots a_n \in \mathds{Q}$.
\begin{enumerate}[label=(\alph*)]
\item Show that if $p(x)$ and $q(x)$ are polynomials then so are $p(x) + q(x)$ and $p(x)q(x)$.

\item Deduce as a corollary that, for all $a, b \in \mathds{Q}$, the linear combination $ap(x)+bq(x)$ of two polynomials
$p(x)$ and $q(x)$ is a polynomial.

\item Show that there exists a polynomial $p_2(x)$ such that $p_2(n) = \sum_{i=0}^n i^2 = 0^2 + 1^2 + \dots + n^2$ for every $n \in \mathds{N}$.

Hint: Note that for every $n \in \mathds{N}$,

$$(n+1)^3 = \sum_{i=0}^n(i+1)^3 - \sum_{i=0}^n i^3$$

\item Show that, for every $k \in \mathds{N}$, there exists a polynomial $p_k(x)$ such that, for all $n \in \mathds{N}$, $p_k(n) = \sum_{i=0}^n i^k = 0^k + 1^k + \dots + n^k$.

Hint: Generalise $$(n + 1)^2 = \sum_{i=0}^n (i+1)^2 - \sum_{i=0}^n i^2$$

and the hint in part (c) above.
\end{enumerate} 

\end{questions}
\end{document}