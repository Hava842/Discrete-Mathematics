\documentclass{exam}
\usepackage[utf8]{inputenc}
\usepackage{amsmath}
\usepackage{hyperref}
\usepackage{enumitem}
\usepackage{ dsfont }
\usepackage{ oz }
 
\begin{document}
\noindent
\large\textbf{Discrete Mathematics 6} \hfill Supervisior: Marton Havasi \\
\normalsize Tryhard solution \hfill 23/02/2018

\begin{questions}

\question Prove that if $A$ and $B$ are sets such that $A \cong \mathbb{R}$ and
$B \cong \mathbb{R}$ then $A \cup B \cong \mathbb{R}$.

Note: I tried to come up with a proof that does not prove the Schröder–Bernstein theorem at the same time but I could not. In set theory, the Schröder–Bernstein theorem states that, if there exist injective functions $f : A \rightarrow B$ and $g : B \rightarrow A$ between the sets $A$ and $B$, then there exists a bijective function $h : A \rightarrow B$. Check out the theorem \href{https://en.wikipedia.org/wiki/Schr%C3%B6der%E2%80%93Bernstein_theorem}{here}.

Assume that $a$ and $b$ are bijections between $A$ and $\mathbb{R}$ and $B$ and $\mathbb{R}$ respectively.

Consider $f(x)=e^x$. $f: \mathbb{R} \rightarrow \mathbb{R}^+$ is a bijection between $\mathbb{R}$ and $\mathbb{R}^+$.

Consider $i: A\cup B \rightarrow \mathbb{R}$:
\[
  i(x) =
  \begin{cases}
    \begin{aligned}
       & a(x)  & \text{ if } x \in A  \\
    
       & b(x) & \text{ otherwise}
    \end{aligned}        
  \end{cases} \,.
\]

$i$ is injective.

Consider $j: \mathbb{R} \rightarrow A\cup B$:
\[
  j(x) =
  \begin{cases}
    \begin{aligned}
       & (f^{-1} \circ a^{-1})(x)  & \text{ if } x > 0  \\
    
       & (f^{-1} \circ b^{-1})(-x) & \text{ if } x < 0 \\
       & a^{-1}(0)  & \text{ otherwise } 
    \end{aligned}        
  \end{cases} \,.
\]

$j$ is injective.

Define the equivalence relations

\[
e_\mathbb{R}(x, y) = \exists n \geq 0. \big((i \circ j)^n(x)=y \wedge (i \circ j)^n(y)=x \big)
\]

\[
e_{A\cup B}(u, v) = \exists n \geq 0. \big((j \circ i)^n(u)=v \wedge (j \circ i)^n(v)=u\big) 
\]

Now we are ready to construct the bijection between $\mathbb{R}$ and $A \cup B$. By definition, the equivalence classes partition their respective sets. Every element is in exactly one equivalence class.

We will construct $h$ by exhibiting a bijection between $[x]_{e_\mathbb{R}}$ and $[j(x)]_{e_{A\cup B}}$ for arbitrary $x$ in $\mathbb{R}$. The resulting relation is a bijection, since it also bijects the elements of $[u]_{e_{A\cup B}} = [j(i(u))]_{e_{A\cup B}}$ and $[i(u)]_{e_\mathbb{R}}$ for arbitrary $u$ in $A \cup B$.

\begin{itemize}
\item Case 1
$[x]_{e_\mathbb{R}}$ has an element $y$ such that $\forall u \in A\cup B. i(u) \neq y$. In this case, $j$ is a bijection between $[x]_{e_\mathbb{R}}$ and $[j(x)]_{e_{A\cup B}}$.
\item Case 2 $[x]_{e_\mathbb{R}}$ does not have an element $y$ such that $\forall u \in A\cup B. i(u) \neq y$. In this case, $i$ is a bijection between $[j(x)]_{e_{A\cup B}}$ and $[x]_{e_\mathbb{R}}$.
\end{itemize}

\end{questions}
\end{document}