\documentclass{exam}
\usepackage[utf8]{inputenc}
 
\usepackage{hyperref}
\usepackage{amsmath}

\begin{document}
\noindent
\large\textbf{Discrete Mathematics 2} \hfill Supervisior: Marton Havasi \\
\normalsize Lectures 4-6 \hfill 11/15/2018

Instructions: Everyone has to do the Core questions. You may skip practice questions under two conditions: 1. You are \textbf{absolutely} certain that you will gain nothing from solving it. That means you went through the solution mentally and you are 100\% sure you know how to solve it for full marks in the exam. 2. You are spending the time you saved to solve the tryhard questions. I trust everyone to make responsible decisions.

\paragraph{Practice questions}
\begin{questions}
\question Exercise sheet 2.2.4

What are $rem(55^2, 79)$, $rem(23^2, 79)$, $rem(23 \cdot 55, 79)$, and $rem(55^{78}, 79)$?


\question Exercise sheet 2.1.4

 Let m be a positive integer.
 
 \begin{enumerate}
\item Prove the associativity of the addition and multiplication operations in $\mathcal{Z}_m$; that is, that for all $i$, $j$, $k$
in $\mathcal{Z}_m$,

$$(i +_m j) +_m k = i +_m (j +_m k) \text{ and } (i \cdot_m j) \cdot_m k = i \cdot_m (j \cdot_m k)$$

\item Prove that the additive inverse of $k$ in $\mathcal{Z}_m$ is $[-k]_m$.
 
 \end{enumerate}

\question Exercise sheet 2.3.2

A decimal (respectively binary) repunit is a natural number whose decimal (respectively binary) representation consists solely of 1’s.
\begin{enumerate}
 \item What are the first three decimal repunits?  And the first three binary ones?
\item  Show that no decimal repunit strictly greater than 1 is square, and that the same holds for binary
repunits.  Is this the case for every base?
Hint:  Use Lemma 26 of the notes.
\end{enumerate} 
\end{questions}

\paragraph{Core questions}
\begin{questions}
\question 2014, Paper 2, Question 7
\href{http://www.cl.cam.ac.uk/teaching/exams/pastpapers/y2014p2q7.pdf}{Link}

\question Consider the statement

$$\forall \text{ natural number }x \text{. } x^{100} - 1 \equiv \prod_{1 \le i \le k} (x-a_i) \mod 101$$

Where $\{a_1, \ldots , a_k \}$ is a finite set of natural numbers.

Find the minimum of $\sum_{1 \le i \le k} a_i$ such that satisfies the statement above.

Did you consider all the edge cases?



\end{questions}

\paragraph{Tryhard questions (please don't cheat by looking them up online)}
\begin{questions}

\question One of Euler's conjectures was disproved in the 1960s by three American mathematicians when they showed there was a positive integer such that $133^5+110^5+84^5+27^5=n^5$. Find the value of ${n}$ without using a calculator.

\question Show that there exists an infinite set $ S$ of positive integers such that for any two distinct elements $ m$ and $ n$ of $ S$, the integers $ 2^m - 3$ and $ 2^n - 3$ are coprime.

\end{questions}

\end{document}
