\documentclass{exam}
\usepackage[utf8]{inputenc}
\usepackage{amsmath}
 
\begin{document}
\noindent
\large\textbf{Discrete Mathematics 1} \hfill Supervisior: Marton Havasi \\
\normalsize Lectures 1-3 \hfill 08/11/2017
\paragraph{Topics}

proof, implication, contrapositive, modus ponens, bi-implication, divisibility, congruence, universal quantification, equality, conjunction, existential quantification, unique existence 

\paragraph{Core questions}
\begin{questions}
\question Prove or disprove the following statements. Clear, step-by-step proofs are expected.
\begin{itemize}
\item Suppose $n$ is a natural number larger than 2, and $n$ is not a prime number. Then $2n + 13$ is not a prime
number.

\item If $x^2+y = 13$ and $y\neq 4$ then $x\neq 3$.

\item For an integer $n$, $n^2$ is even if and only if $n$ is even.

\item For all real numbers $x$ and $y$ there is a real number $z$ such that $x + z = y - z$.

\item For all integers $x$ and $y$ there is an integer $z$ such that $x + z = y - z$.

\item For all integers $m$ and $n$, if $mn$ is even, then either $m$ is even or $n$ is even.

\item $10|1526^{19}+2^{58}$
\end{itemize}

\question Find all $p$ prime numbers, such that $\frac{p^2-1}{p-1}$ is also prime. 

\question Let $P(m)$ be a statement for $m$ ranging over the natural numbers, and consider the derived statement $$P^\#(m)=(\forall \text{ natural number } k \text{ . } 0 \leq k \leq m \implies P(k))$$ 

again for m ranging over the natural numbers.

\begin{itemize}
\item Show that for all natural numbers $l$, $P^\#(l)\implies P(l)$
\item Prove by exhibiting a counter-example that $P(n)\implies P^\#(n)$ does not hold.
\item Prove or disprove:
\begin{itemize}
\item $P^\#(0) \iff P(0)$
\item $\forall \text{ natural number } n. (P^\#(n) \implies P^\#(n+1)) \iff (P^\#(n) \implies P(n+1)) $
\item $(\forall \text{ natural number } m.P^\#(m) ) \iff (\forall \text{ natural number } m.P(m) )$
\end{itemize}
\end{itemize}

\question Prove that for all integers d, k, l, m, n,
\begin{itemize}
\item $d | m \wedge d | n \implies d | (m + n)$
\item $ d | m \implies d | k m$
\item $ d | m \wedge d | n \implies d | (k m + l n)$
\end{itemize} 

\end{questions}

\paragraph{Tryhard questions (entirely optional, can be difficult)}
\begin{questions}
\question Find all natural numbers $n$, such that $n^3-27$ is a prime.

\question Prove that there are infinitely many natural numbers $n$, such that $4n+3$ is prime.
\end{questions}

\paragraph{Survey Questions}
\begin{questions}
\question How long did it take to complete the core questions?
 
\question How do you rate your understanding of the topics of this week's supervision? (select one or more)

\begin{itemize}
\item
I have little clue
\item 
I understand some of the topics
\item
I understand most of the topics
\item
Take me to the exam hall
\end{itemize}
\end{questions}
\end{document}